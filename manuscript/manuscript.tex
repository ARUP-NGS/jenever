\documentclass[]{article}
\usepackage{amsmath}
\usepackage{graphicx}
\usepackage{authblk}
\graphicspath{ {./images/} }
\linespread{1.15}
\textwidth=380pt


\title{Variant calling with transformers}
\author[1]{Brendan O'Fallon}
\author[1]{Ashini Bolia}
\author[1]{Jacob Durtschi}
\author[1]{Luobin Yang}

\affil[1]{ARUP Institute for Clinical and Experimental Pathology, Salt Lake City, UT}
\date{}
\begin{document}

\maketitle

\begin{abstract}
	Detection of germline variants in next-generation sequencing data is an essential element of modern genomics. Variant detection tools typically rely on a variety of statistical algorithms and heuristic techniques to identify variants. Here we describe a new approach that replaces the handcrafted statistical methods with a single, end-to-end deep learning model. Our model frames variant detection as a sequence-to-sequence modeling task, and employs a transformer-based architecture to translate alignment columns into two predicted haplotype sequences. We demonstrate that this method predicts variant genotypes accurately, phases nearby variants correctly, and has the ability to realign complex and ambiguous read mappings to produce accurate haplotype predictions. 

\end{abstract}



\section{Introduction}

Accurate reconstruction of the DNA sequence in a sample is a fundamental component of modern genomics analysis. Next Generation Sequencing (NGS) workflows typically begin by aligning sequenced fragments to a reference genome, optionally performing some alignment post-processing steps such as identification of duplicated fragments, and then identifying sequence variants relative to the reference genome. The variant identification step is challenging for many reasons, including technical artifacts in the sequencing process, improper alignments due to sequence homology or low sequence complexity, and the stochastic nature of the library preparation and amplification steps [citation here would be nice]. 

All variant discovery tools must address two challenges. First, they must identify candidate alleles in a region. Then, given a set of allele candidates, they must assess the likelihood of each candidate (or pairs of candidates in the diploid case), to determine the most likely alleles. Early callers, such as samtools / mpileup (Li et al. 2009) and the UnifiedGenotyper tool from the Genome Analysis ToolKit (GATK, De Pristo et al. 2011) rely on the read aligner to generate candidate allele candidates, and utilize ad-hoc heuristics and thresholds to determine likely alleles. Later tools (e.g. Poplin et al. 2018, Kim et al. 2018, Cooke et al. 2021) incorporated local re-alignment of reads in each region of interest, resulting in a significant improvement in the precision and specificity of variant calls. For instance, the HaplotypeCaller tool identifies candidate haplotypes by constructing de Bruijn graphs from k-mers present in the reads, and assesses haplotype likelihoods with a pair Hidden Markov Model (HMM). 

More recent tools have incorporated elements of deep learning into the allele likelihood calculation. DeepVariant (Poplin et al. 2018) builds on the HaplotypeCaller approach, but adds a Convolutional Neural Net (CNN) to classify variants as true or false positive detections. HELLO (Ramachandran et al. 2020), designed to work on hybrid short- and long-read datasets, employs a mixture-of-experts approach with separate 1-dimensional convolutions across the read and allele dimensions of the input. 

Here, we explore a new approach to variant detection that replaces the bespoke statistical techniques used in previous tools with a single, end-to-end machine learning model. We recast variant detection as a sequence-to-sequence translation problem in which the input sequence is the series of bases aligning to a single genomic location, and the output sequences are the two predicted haplotypes. The sequence-to-sequence translation is performed by a transformer-based deep neural network (Vaswani et al. 2017). We demonstrate that this approach identifies real genomic variants with sensitivity equivalent to current top-performing callers, albeit with slightly lower specificity. 



\section{Methods}

\subsection{Model architecture}

Our model consists of a transformer-based encoder and simple, fully connected decoder that produces two output sequences, one for each predicted haplotype. The encoder component is an unmodified transformer with GeLU activations as implemented in PyTorch 1.10 (Paszke et al. 2019). Input tokens are the collection of bases that align to a given reference position across reads, with some modifications described below. We add an additional fully connected layer prior to the transformer encoders which embeds the encoded basecalls in $d$ dimensions, where $d=12$ for the analyses here. The embedded basecalls are 'flattened' along the read dimension, producing an input token with size $dr$ where $r$ is the number of reads. Instead of the typical 1-dimensional positional encoding, we employed a 2-dimensional encoding, enabling the model to retain information about which read contained a given base across input tokens. 

The decoder consists of a single fully-connected layer followed by two additional fully connected components that each generate a single haplotype. Predictions are discrete probability distributions over the four possible bases at each predicted output position. The decoder is of fixed size, and each predicted haplotype contains the same number of predicted bases as the number of positions $g$ in the input. 


\subsection{NGS data}

We obtained training data from 17 whole genomes sequenced from 5 Genome-in-a-Bottle cell lines. Eight samples were prepared with Illumina Nextera DNA Flex kit and the remainder were prepared with the Illumina TruSeq PCR-free kit. All samples were sequenced on an Illumina NovaSeq 6000 instrument in 2x150 mode, to an approximate read depth of 50. After conversion to fastq, the sequenced reads were aligned to human reference genome GRCh37 with the GEM-mapper (v3, Marco-Sola et al. 2012) and were sorted and converted to CRAM format with samtools version 1.9 (Li et al. 2009). No additional refinements, such as duplicate read marking, base-quality score recalibration, or indel realignment were performed.

\subsection{Training data}

Training tensors were generated by selecting regions 150bp in width from a list of preselected regions, in a manner described below. For each region, all reads overlapping the window were obtained from the corresponding BAM/ CRAM file. If more than a specified maximum number of reads were found (100 for all analyses here), then reads were downsampled to the desired size.  Reads were sorted by the reference coordinate of the first aligned base. For each aligned base in the selected reads nine features were encoded; the first four were the one-hot encoded base call, followed by base quality, two flags indicating if base `consumed' reference base (i.e. was not an insertion) or consumed a sequence read base (was not a deletion), and additional flags indicating sequence read direction and clipping status.  No 'gap' tokens or other special handling was performed for insertions or deletions. In addition to sequenced reads, the first row in each encoded region was the reference sequence. For this special row we inserted a base quality of 100 for every position and did not set any of the other flags. Resulting tensors had dimension $[g, r, 9]$, where $g$ represents genomic position, $r$ represents read, and 9 is the number of features. 

Target / label haplotype sequences were produced by obtaining truth variants from the Genome-in-a-Bottle VCF files for each sample and inserting the variants into the reference sequence. Two sequences were generated for each region, one representing each haplotype. In regions where phasing of the variants was ambiguous, the reads in the sequenced sample were examined to determine phase status. Briefly, all possible genotypes (pairs of haplotype sequences) were generated and reads were aligned via Smith-Waterman to the possible haplotypes, and the highest scoring genotypes selected as the most likely phasing. This phasing procedure procedure was only attempted for variants less than 100bp apart, otherwise the region was discarded.

To select regions to include for training data, we developed a scheme to sample regions in a biased manner, prioritizing regions containing variants and, especially, regions with multiple or complex variants. Regions of the reference genome overlapping the high-confidence regions from Genome-in-a-Bottle were subdivided into 150bp nonoverlapping windows, and these regions labelled according to the presence of variants. Separate labels were generated for regions conataining a single SNV, deletion, or insertion, as well as regions containing multiple insertion-deletions, or those containing variants intersecting  low-complexity regions. We additionally included `true negative' regions where no known variant was present. The total number of regions generated per sample was 325,000, yielding 5.53M total training regions across our 17 samples.

For all analyses described here, we obtain training data only from the human autosomes 1-20, and hold out chromosomes 21 and 22 for model evaluation. 

\subsection{Loss function}

We use a standard cross-entropy loss function to compare predicted haplotypes to known haplotype sequences for each region, and simply compare each prediction at window index (position) $i$ to the true base at index $i$.  While we obtain satisfactory results with this approach, two issues were apparent. First, predicted haplotype sequences may have inserted or deleted bases relative to the true haplotypes. Because incorrect insertions and deletions cause mismatches at every downstream (higher index) position, such mistakes are penalized more heavily when they occur at lower indexes than higher indexes. Thus the model learns to generate high quality results for the first portion of the window where inaccuracies are punished severely, and is less concerned with accuarcy at high indexes. 

A second complexity arises due to ambiguity regarding which haplotype should contain a variant. The two haplotypes present in a sample have no inherent order. In training data variants are assigned arbitrarily to haplotypes, and this randomness in haplotype assignment means model cannot learn to predict which haplotype should contain a given variant. To combat this ambiguity, we calculate loss in both possible configurations - predicted haplotype 0 with training haplotype 0 and 1 with 1, as well as predicted haplotype 0 with training haplotype 1 and 1 with 0. We select the configuration with the lowest loss and backpropogate gradients only for that configuration. 


\subsection{Variant detection}

Given an alignment file in BAM or CRAM format, we first identify regions where a potential variant might exist, and then run forward passes of the model using a sliding window over the region. Any genomic position in which at least three reads contain a base that differs from the reference or an indel are flagged as potentially containing a variant. Positions closer than 100bp are merged into a single region. 

Given a region containing suspected variants, we perform multiple overlapping forward passes of the model with step size $k$. On each forward pass the model produces two predicted haplotypes. Each haplotype is aligned via Smith-Waterman to the reference sequence, and any mismatching positions are converted to variant calls. We record the number of windows in which each variant was detected

\section{Results}

\subsection{Genotyping}
 - Are we getting the right variant genotypes

\subsection{Phasing accuracy}
 - look at nearby variants, see if they are phased correctly compared to GIAB information

\subsection{Complex regions}
 - Not sure how to define these, but ideally we'd look for variants in regions where the read mapper had lots of overlapping insertion / deletions in reads and/or softclipping

\subsection{Use of local context}
 - Also ambiguous, but anecdotally when looking at variant calls, it seems apparent that variants that otherwise look real have lower quality scores when there's a lot of junk around. That's good, since such variants are more likely To
 be false positives, and that means the model is to some extent looking at local context to modify prediction accuracy. Which is cool. 


\subsection{Variant detection accuracy}

 - Tabulate Hap.py results across a few common stratifications, compare to other callers




\subsection{Model size experiments}

 small (5M) medium (10M), large (25M) pretty big (50M)

 
 Ablate the 2D positional encoding.

 Ablate reference as read 0 
 
 \subsection{Phasing accuracy}



 \section{Discussion}

We describe a new approach to the problem of detecting sequence variants in next-generation sequencing (NGS) data based on a single, end-to-end deep learning model. Our approach envisions variant detection as a sequence-to-sequence modeling problem, akin to language translation, and leverages the successful transformer architecture to accomplish the task. In contrast to other variant detection methods (...), our approach does not rely on handcrafted statistical techniques such as de Bruijn graphs, hidden Markov models, or heuristic thresholding and instead learns to construct accurate haplotypes directly from aligned NGS reads. 
 
 Convolutions - do they make sense? 

 Loss functions, there's probably something better. The Smith Waterman loss sounds really cool, but didn't lead to high sensitivity in our hands. 
 
 \section{Availability}
 
 Source code for is available via git at https://github.com/ARUP-NGS/somewhere
 
 \section{References}
 
 \vspace{8pt}
 Conrad, Donald F., et al. "Origins and functional impact of copy number variation in the human genome." Nature 464.7289 (2010): 704.

\vspace{8pt}
Cooke, Daniel P., David C. Wedge, and Gerton Lunter. "A unified haplotype-based method for accurate and comprehensive variant calling." Nature biotechnology 39.7 (2021): 885-892.

\vspace{8pt}
DePristo, Mark A., et al. "A framework for variation discovery and genotyping using next-generation DNA sequencing data." Nature genetics 43.5 (2011): 491-498.


\vspace{8pt}
Kim, Sangtae, et al. "Strelka2: fast and accurate calling of germline and somatic variants." Nature methods 15.8 (2018): 591-594.

\vspace{8pt}
Li, Heng, et al. "The sequence alignment/map format and SAMtools." Bioinformatics 25.16 (2009): 2078-2079.

\vspace{8pt}
Li, Heng, and Richard Durbin. "Fast and accurate short read alignment with Burrows–Wheeler transform." Bioinformatics 25.14 (2009): 1754-1760.

\vspace{8pt}
Marco-Sola S., Sammeth M., Guigó R., Ribeca P. "The GEM mapper: fast, accurate and versatile alignment by filtration". Nat Methods. (2012);9(12):1185-1188. doi:10.1038/nmeth.2221

 
\vspace{8pt}
Paszke, A., et al. "PyTorch: An Imperative Style, High-Performance Deep Learning Library." In Advances in Neural Information Processing Systems 32 (2019):8024–8035. Curran Associates, Inc. Retrieved from http://papers.neurips.cc/paper/9015-pytorch-an-imperative-style-high-performance-deep-learning-library.pdf


\vspace{8pt}
Poplin, Ryan, et al. "A universal SNP and small-indel variant caller using deep neural networks." Nature biotechnology 36.10 (2018): 983-987.

\vspace{8pt}
Poplin, Ryan, et al. "Scaling accurate genetic variant discovery to tens of thousands of samples." BioRxiv (2018): 201178.

\vspace{8pt}
Ramachandran, Anand, et al. "HELLO: A hybrid variant calling approach." bioRxiv (2020).
 

\vspace{8pt}
Vaswani, Ashish, et al. "Attention is all you need." Advances in neural information processing systems 30 (2017).


\end{document}


